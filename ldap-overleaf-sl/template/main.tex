\documentclass{rapportECL2024}

\addbibresource{biblio.bib}

%--------------------------------------

\titre{Titre}
\soustitre{Sous-titre}

\enseignant{Prénom \textsc{Nom} \\
            Prénom \textsc{Nom} }

\eleves{Prénom \textsc{Nom} \\
	Prénom \textsc{Nom} \\ 
	Prénom \textsc{Nom} }

%--------------------------------------

\begin{document}

\fairepagedegarde
\fairetabledesmatieres

%--------------------------------------

\section{Section 1}
\lipsum[1]


\subsection{Sous-section 1}
\lipsum[2]

\subsection{Sous-section 2}
\lipsum[3]

\subsubsection{Sous-sous-section 1}
\lipsum[4]

\section{Section 2}
\lipsum[5]

\section{Des commandes utiles}

\textbf{Insérer une image}
\begin{figure}[H]
    \centering
    \includegraphics[width=0.5\textwidth]{logos/logo.png}
    \caption{Insérer une image.}
    \label{fig:logo}
\end{figure}

ou la version raccourcie avec \texttt{\textbackslash insererfigure} : \insererfigure{logos/logo.png}{0.5\textwidth}{Insérer une image avec une commande.}{logo_raccourci}


\textbf{Citer une figure} avec \texttt{\textbackslash ref} : \ref{fig:logo_raccourci}.\\


\textbf{Insérer des expressions mathématiques} en bloc :
\begin{equation*}
    \int_{-\infty}^{\infty} e^{-x^2} \, \text{d}x = \sqrt{\pi}
\end{equation*}

ou bien en \textit{inline} : $\int_{-\infty}^{\infty} e^{-x^2} \, \text{d}x = \sqrt{\pi}$.\\


\textbf{Insérer une série de calculs}
\begin{align*}
    x &= 0.999\ldots \\
    10x &= 9.999\ldots \\
    10x - x &= 9.999\ldots - 0.999\ldots \\
    9x &= 9 \\
    x &= 1 \\
    0.999\ldots &= 1
\end{align*}


\textbf{Insérer une liste}
\begin{itemize}
    \item Premier niveau
    \begin{itemize}
        \item Deuxième niveau
        \item Un autre élément au deuxième niveau
    \end{itemize}
    \item Un autre élément au premier niveau\\
\end{itemize}


\textbf{Insérer un tableau simple} \\
\begin{table}[H]
    \centering
    \begin{tabular}{|c|c|c|}
        \hline
        A & B & C \\
        \hline
        1 & 2 & 3 \\
        4 & 5 & 6 \\
        \hline
    \end{tabular}
    \caption{Un tableau simple.}
    \label{tab:tableau_simple}
\end{table}


\textbf{Insérer du code} : \texttt{print("Hello, World!")}.\\


\textbf{Citer une source} depuis \texttt{biblio.bib} avec \texttt{\textbackslash cite} : \cite{exemple_de_source}.\\


\textbf{Afficher une bibliographie} rapidement avec \texttt{\textbackslash insererbiblio}.

\insererbiblio 

\end{document}